\documentclass[twocolumns,12pt]{IEEEtran}
%\usepackage[spanish]{babel}
\usepackage[utf8]{inputenc}
\usepackage{amsmath, amsthm, amsfonts}
\usepackage{graphicx}
\usepackage{cite}
\usepackage{listings}
\usepackage{tabularx}
\usepackage{multirow}
\usepackage{colortbl}
\renewcommand{\refname}{Referencias}
\usepackage{subfig}
\usepackage{booktabs}
\usepackage{cite}
\usepackage{float}
\usepackage{multirow}
\renewcommand{\citedash}{ -- } 
\providecommand{\abs}[1]{\lvert#1\rvert}


\begin{document}
\renewcommand{\tablename}{\small{\textsc{Tabla}}}

\title{Práctica 4: Amplificador par direfencial}
\author{Esteban Ladino F. - Mateo Alvarez L. - Edward Alonso Villamil A.\\ 
\{eladinof, maalvarezle, eavillamila\}@unal.edu.co\\ Universidad Nacional de Colombia\\
Facultad de Ingeniería\\
Electrónica Análoga II}
\maketitle


\renewcommand{\leftmark}{Práctica 4: Amplificador par direfencial., E. Ladino, M. Alvarez y E. Villamil, noviembre 17 del 2019}  

\maketitle


\begin{abstract}

\end{abstract}

\renewcommand{\abstractname}{Resumen}
\begin{abstract}

\end{abstract}

\renewcommand{\IEEEkeywordsname}{Palabras Clave}
\begin{IEEEkeywords} 

\end{IEEEkeywords}

\IEEEpeerreviewmaketitle

\section{Introducción}

\IEEEPARstart{U}{}
\section{Trabajo Previo}

\section{Simulación}

\section{Trabajo en el Laboratorio}



\section{Trabajo en casa}


\begin{table}[H]
    \centering
    \caption{\textsc{\footnotesize Puntos de polarización de M3. Donde E.R error relativo y E.A error absoluto.}}
    \label{Tabla 1}
    \begin{tabular}{@{}ccccc@{}}
    \toprule
    \toprule
    Magnitud  &Sim. [V] & Med. [V] &E.R [V] &E.A [V]\\ \midrule

  VDS    &  1.503  &  1.501  &  0.1331  &  0.0020  \\


  VGS    &  1.521  &  1.578  &  -3.748  &  -0.0570  \\


  VGD    &  0.018  &  0.073  &  -301.1  &  -0.0548  \\


  V(RD)   &  1.689  &  1.739  &  -2.960  &  -0.0500  \\


    
     \bottomrule \bottomrule
    \end{tabular}
\end{table}


%\begin{figure}[H]
%\centering
%\includegraphics[width=1\linewidth]{Rin.png}
%\caption{Cálculo de la resistencia interna.}
%\label{Rin}
%\end{figure}


\section{Conclusiones}
\begin{itemize}
\item 

\end{itemize}



\begin{thebibliography}{1}

\bibitem{1} B. Razavi, Fundamentals of Microelectronics, Current Mirrors, ED. 2. Los Angeles: University of California, 2014, 
página 421-422.  

\end{thebibliography}


\end{document}
